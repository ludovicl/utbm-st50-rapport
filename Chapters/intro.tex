%!TEX root =/Users/ludovicl/Dropbox/Cours/UTBM/P15/RapportStage/main.tex

\newpage

\section*{Introduction}

Dans le cadre de ma formation d’ingénieur à l’Université de Technologie de Belfort Montbéliard j’ai dû effectuer un stage de vingt-six semaines à l’issue de la dernière année de mon cycle d’ingénieur. Ce stage m'a permis de me familiariser d'avantage avec le monde de l’entreprise et de mettre en pratique les connaissances acquises tout le long de mon cursus scolaire.
\\
J’ai effectué mon stage au sein de de la startup Tech4Team située à Paris.
\\ 

\indent Durant ce stage j'ai eu l'occasion de travailler avec des technologies comme Ruby on Rails, Python, PostgreSQL, Riak, différentes API,...
\\ \\
Dans la première partie de ce rapport, je présenterai l’entreprise Tech4Team en faisant un bref aperçu des logiciels développés et de l'organisation de l'équipe.
Ensuite j'aborderais les aspects plus technique avec le travail que j'ai eu à effecteur et les problèmes que j'ai rencontrés et eu a traiter.

Je terminerai pas une conclusion, les résultats et ferait le bilan de ce stage.


%\\
%Dans la deuxième partie j’évoquerai la constitution du pôle R&D de Parkeon, les interactions entre les différentes équipes ainsi que les activités et les méthodes de développement de l’équipe lecteur.
%
%
%
%Par la suite je développerai mon sujet de stage en indiquant comment j’ai acquis des compétences sur les logiciels en cours de développement par l’équipe lecteur. J’indiquerai aussi mon rôle dans les tests de la nouvelle sécurité d’échange des données.
%\\
%Pour terminer, je ferai un point sur les résultats que j’ai obtenus à l’issu de ces vingt-quatre semaines de stage et je présenterai ce que mon travail apporte à l’entreprise.

\newpage
\section*{Remerciements}
Je tiens à remercier Kevin Vitoz et Ludovic Bordes, qui m'ont fait confiance, m'ont bien accueilli et m'ont fait découvrir leur entreprise.
\\ \\ 
Je remercie également Marwan Rabbaa pour ses précieux conseils techniques, et son soutien.
%!TEX root =/Users/ludovicl/Dropbox/Cours/UTBM/P15/RapportStage/main.tex
\chapter{Résultats et conclusion}

\section{Résultats}
Au terme de mon stage nous avons 15 clients en bêta test, la plupart des données sont encore importées à la main avec des fichiers CSV mais plus l'entreprise grandie et gagne en notoriété et plus nous avons accès aux API des logiciels de billetteries.
\\

D'ici quelques mois nous pourrons faire du temps réel sur la synchronisation des données et appliquer les modèles de prévisions à la volée.

\section{Conclusion}
Ce stage m'a permis de me familiariser davantage avec le monde de l'entreprise et cela de plusieurs manières :
	\begin{itemize} 
		\item[\textbullet] J'ai pu mettre en œuvre les connaissances acquises dans le domaine des bases de données dans un cadre professionnel
		\item[\textbullet]Je me suis initié au management en devant m'occuper de stagiaires
		\item[\textbullet]J'ai découvert le développement d'applications de type SaaS et les contraintes techniques que cela implique notamment au niveau des serveurs. 
	\end{itemize} 
\leavevmode \\
Après un premier stage effectué dans une entreprise de taille intermédiaire, je peux maintenant comparer ces deux mondes professionnels.
\\ \\
\textbf{Les entreprises de tailles intermédiaires}
\\
L'avantage que je trouve à travailler en tant que stagiaire dans ce genre d'entreprise est l'encadrement.
Les ingénieurs de ce type d'entreprise ont plus de disponibilités et peuvent nous guider dans l'exécution de certaines tâches. On peut s'appuyer sur leurs conseils. 

On sait exactement où l'on va, le travaille que l'on a faire est les collègues peuvent prendre du temps pour nous aider. Par contre on est juste y engrenage dans une machine beaucoup plus complexe.
\\ \\
\textbf{Les start-ups}
\\
L'encadrement est beaucoup moins présent, le travail à faire évolue au cours des semaines et des besoins des clients. Le nombre d'employés étant moindre le stagiaire a plus de responsabilités, il faut prendre des décisions, implémenter des fonctionnalités rapidement, on se forme on apprend sur le tas, c'est très formateur \\
Le seul risque est de prendre des mauvaises habitudes, mais avec tous les logiciels d'analyse de codes et les ressources à notre disposition on s'en sort.
\\
\\
\\

Mon stage s'est terminé le 31 juillet, l'entreprise m'a proposé un CDI que j'ai accepté. Mon rôle dans l'entreprise est de m'occupé de de la partie back-end et de la conception des bases de données.   

 
%!TEX root =/Users/ludovicl/Dropbox/Cours/UTBM/P15/RapportStage/main.tex
\chapter{Mise en situation et description des activités du stagiaire}


\section{Importation manuel de données billetteries des clients dans ArenaPricing}

L'une de mes premières taches en arrivant dans l'entreprise est de créer un logiciel permettant à Tech4Team et à certains clients de facilement insérer dans notre base de données un fichier CSV\footnote{Comma-separated values, connu sous le sigle CSV, est un format informatique ouvert représentant des données tabulaires sous forme de valeurs séparées par des virgules.} contenant des données billetteries. A tèrme le but est de d'intégrer le logiciel les API des logiciels de billetterie des clients.

\subsection{Le besoin}

Le besoin peut donc être visualisé par le schéma \ref{interface_upload} page \pageref{interface_upload}. Tech4Team ou un client veut insérer des nouveaux tickets dans la base de données Arenapricing. Il sélectionne alors un fichier CSV à la main depuis l'interface web, le site analyse le fichier, extrait les informations importantes et les inserts en base de donnes. Toute la partie d'analyse et d'insertion doit être invisible du point de vue de l'utilisateur.
\\ \\
Le logiciel d'upload doit également être capable de s'interfacer avec des API.

\begin{center}
\includegraphics[scale=0.6]{images/datafit.png}
\captionof{figure}{Principe de l'interface d'upload}
\label{interface_upload}
\end{center}


\subsection{Première réalisation du front d'importation}

Pour réaliser ce site web d'importation nous avons décidé après discussion avec le lead developper d'utiliser le langage Python. Ce langage permet de développer rapidement est dispose d'énormément de modules s'ajoutant aux fonctions de base. Il est ainsi aisé de parser un fichier CSV. J'ai également utilisé le microframework Flask pour créer les pages web en elle même et servir de serveur web. Pour le design du site, j'ai utilisé le framework Zurb Foundation afin d'avoir une identité visuelle cohérente. 

\begin{center}
\includegraphics[scale=0.6]{images/datafit2.png}
\captionof{figure}{Téchnologies utilisées pour le front}
\label{interface_upload_tech}
\end{center}

\subsubsection{L'interface homme machine obtenu}

Comme nous pouvons le voir sur l'impression d'écran \ref{front_upload} page \pageref{front_upload} l'interface est ermet à l'utilisateur d'envoyer un fichier CSV préalablement sélectionné.
En plus du nom de l'organisation, nous affichons le type d'événement correspondant au fichier envoyé ainsi que le logiciel de billetterie d'où provient l'export. \\

\begin{center}
\includegraphics[scale=0.6]{images/front1.png}
\captionof{figure}{Front d'upload d'un fichier CSV}
\label{front_upload}
\end{center}

\subsubsection{Parsing des données en Python}
Pour parser les fichiers CSV j'utilise le module natif CSV.
\\


\lstset{style=custompython}
\begin{lstlisting}
with open(file_path, 'r', encoding=charset['encoding']) as f:

	#read csv file with csv module
	dict_csv = csv.DictReader(f, delimiter=';', quoting=csv.QUOTE_ALL)
	
	#use strategy design pattern with Secutix object
	secutix = ParsingStrategyContext(Secutix())	
	
	#parse dictionay and send organization name, event date, export date
	secutix.parse_file(dict_csv, o_name, e_date, export_date)

#ask object to send data to the ruby. e_type is event type 
reason = secutix.upload_data(e_type)
return reason
\end{lstlisting}
\captionof{lstlisting}{Code analysant le fichier CSV}

\leavevmode \\
La fonction open permet d'ouvrir le fichier $file\_path$, ensuite je récupère le dictionnaire associé à au fichier CSV avec $csv.DictReader$. La méthode $parse\_file$ permet de générer le dictionnaire avec les informations des tickets/achats comme le montre la figure \ref{dict_csv} page \pageref{dict_csv}, nous obtenons finalement une liste de dictionnaire ou chaque entrée de la liste est une ligne du fichier CSV. 

\begin{center}
\includegraphics[scale=0.45]{Images/dict_tickets.png}
\captionof{figure}{Dictionnaire récupéré pour une ligne du fichier CSV}
\label{dict_csv}
\end{center}


D'un point de vue architecture de l'application j'utilise le design pattern strategy. Ce design pattern est utile lors qu'un objet peut effectuer plusieurs traitements différents, dépendant d'une variable ou d'un état.

L'implémentation de ce design pattern peut être visualisé sur le diagramme figure \ref{stategy_pattern} page \pageref{dict_csv}.

\begin{center}
\includegraphics[scale=0.75]{Images/StrategyPattern.png}
\captionof{figure}{Design pattern strategy}
\label{stategy_pattern}
\end{center}



\subsubsection{Problèmatique posée par ce logiciel}
Après avoir testé le logiciel d'importation avec de nombreux clients il nous apparait que les fichiers fournis ne sont pas génériques. 
Les noms des colonnes ne sont pas les même en fonction des exports alors que le logiciel de billéterie est indentique et dans certaines cas le changement apparait pour un même client d'une saison à l'autre. 
On se rend donc rapidement compte que nous ne pouvons pas créer un script unique par logiciel de billetterie.

\subsection{Deuxième itération pour le logiciel d'import des fichiers}
Nous décidons de repenser notre logiciel pour que le client désirant envoyer un fichier CSV puisse spécifier sur l'interface les champs qu'il souhaite importer.

Comme le montre l'image \ref{draft_front_import2} page \pageref{draft_front_import2} du premier draft que j'ai réalisé, le client doit maintenant faire un travail d'association entre les champs présents dans le fichier CSV leur correspondance réel. 

\begin{center}
\includegraphics[scale=0.55]{images/front3.png}
\captionof{figure}{Premier draft du logiciel d'importation 2.0}
\label{draft_front_import2}
\end{center}


\subsubsection{Création de l'IHM}
Pour réaliser l'interface d'upload j'ai utilisé le framework Ruby on rails afin pouvoir intégrer facilement la page d'upload au reste du site.

\begin{center}
\includegraphics[scale=0.37]{images/final_front2.png}
\captionof{figure}{Logiciel d'importation des clients}
\label{final_front_import2}
\end{center}

J'obtient finalement le front-end \ref{final_front_import2} page \pageref{final_front_import2}, comme nous pouvons le voir avec le fichier CSV de test les champs sont mapés ensemble. Si la donné n'est pas présente une insertion vide se fait en base de données. Les champs marqué avec une étoile sont indispensable. Certains champs comme le nom de l'événement et la date peuvent être inscrit à la main si ils ne sont pas fournis dans le fichier CSV.


\subsubsection{L'importation des fichiers CSV}

Afin d'importer les données dans le loficiel de billéterie j'utilise l'ORM ActiveRecord.

\lstset{style=customruby}
\begin{lstlisting}
begin
  country = Pricing::Country.find_by!(name: customer_country_to_ins)
rescue ActiveRecord::RecordNotFound 
  country = Pricing::Country.new(name: customer_country_to_ins)
end
country.save
begin
  department = Pricing::Department.find_by!(name: customer_zip_to_ins, country: country)
rescue ActiveRecord::RecordNotFound
  department = Pricing::Department.new(name: customer_zip_to_ins,  country: country)
end
department.save
begin
  city = Pricing::City.find_by!(name: customer_city_to_ins, department: department)
rescue ActiveRecord::RecordNotFound
  city = Pricing::City.new(name: customer_city_to_ins,  department: department)
end
city.save
\end{lstlisting}
\captionof{lstlisting}{Exemple de code insérant une adresse}
\leavevmode \

Avant d'insérer une donnée je vérifie si elle si elle est présente, si ce n'est pas le cas je rescue l'exception et j'insère la donnée. 
L'ORM ActiveRecord permet de s'astreindre des contrainte habituels du SQL, lorsque je fais coutry: country lors d'une intertion de department l'ORM comprend qu'il doit lier country comme clé étrangère de department.

La majorité des des clients nous fournissent des fichiers CSV. Ces fichiers sont extraits à partir du logiciel de billéterie utilisé par le client.


\subsection{Les API des logiciels billéteries}

















